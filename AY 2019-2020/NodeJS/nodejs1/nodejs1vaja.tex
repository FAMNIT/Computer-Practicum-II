\documentclass[a4paper,10pt]{article}
\usepackage[utf8]{inputenc}
\usepackage[slovene]{babel}
\usepackage{fullpage}
\usepackage{enumitem}
\usepackage{hyperref}

\hypersetup{
    colorlinks=true,
    linkcolor=blue,
    filecolor=magenta,      
    urlcolor=cyan,
}

\title{Računalniški praktikum 2}
\author{Node.js - samostojna vaja 1}
\date{8. maj 2020}

\begin{document}
\maketitle

\noindent
Na predavanjih smo napisali strežniško skripto za zelo enostaven klepet. Za samostojno vajo to skripto dopolnite tako, da bo omogočala prijavo s poljubnim vzdevkom (angl. \emph{nickname}). Pred vsekim objavljenim sporočilom se tako izpiše še vzdevek avtorja sporočila. Poleg tega omejite izpis sporočil na zadnjih 20. Zgled takšnega klepeta lahko preizkusite na naslovu \url{http://2.56.215.150:8080/chat.html}.

\medskip\noindent
Strežniška skripta razume dva parametra:
\begin{itemize}[noitemsep]
\item \texttt{nick} - avtor sporočila,
\item \texttt{msg} - sporočilo.
\end{itemize}
Zgled pravilnega URL-ja:
$$
\texttt{http://2.56.215.150:8080/chat.html?nick=Domen\&msg=Hello people!}
$$

\medskip\noindent
Na podlagi podanega URL-ja prikaže eno izmed treh enostavnih HTML strani:
\begin{enumerate}
\item Dani URL mora zahtevati stran \texttt{chat.html}, sicer strežnik izpiše:
\begin{verbatim}
<!DOCTYPE html>
<html>
<head>
<meta charset="UTF-8">
<title>RP2 Chat</title>
</head>
<body>
<div>Page not found!</div>
</body>
</html>
\end{verbatim}

\item Če parameter \texttt{nick} ni nastavljen oz. če je prazen, strežnik izpiše prijavno stran:
\begin{verbatim}
<!DOCTYPE html>
<html>
<head>
<meta charset="UTF-8">
<title>RP2 Chat</title>
</head>
<body>
<form>
<div>Nickname:</div>
<div><input type="text" name="nick" autofocus>&nbsp;
<button>Chat!</button></div>
</form>
</body>
</html>
\end{verbatim}

\newpage
\item Če je parameter \texttt{nick} podan, strežnik izpiše zadnjih 20 sporočil. Če je vseh sporočil manj kot 20, preostali prostor zapolni s praznimi vrsticami:
\begin{verbatim}
<!DOCTYPE html>
<html>
<head>
<meta charset="UTF-8">
<title>RP2 Chat</title>
</head>
<body>
<div>&nbsp;</div>
<div>&nbsp;</div>
<div>&nbsp;</div>
<div>&nbsp;</div>
<div>&nbsp;</div>
<div>&nbsp;</div>
<div>&nbsp;</div>
<div>&nbsp;</div>
<div>&nbsp;</div>
<div>&nbsp;</div>
<div>&nbsp;</div>
<div>&nbsp;</div>
<div>&nbsp;</div>
<div>&nbsp;</div>
<div>&nbsp;</div>
<div>&nbsp;</div>
<div>&nbsp;</div>
<div>&nbsp;</div>
<div>Domen: Hello people!</div>
<div>Domen: How are you?</div>
<form>
<input type="hidden" name="nick" value="Domen">
<div>
Message:&nbsp;<input type="text" name="msg" autofocus>&nbsp;
<button>Send</button>
</div>
</form>
<div><a href="/chat.html">Log out</a></div>
</body>
</html>
\end{verbatim}
Strežnik vzdevek prejme preko parametra \texttt{nick} in ga vrne že predizpolnjenega v obrazcu (glej vnosno polje z imenom \texttt{nick}). To vnosno polje je uporabniku skrito (\texttt{type="hidden"}), zato ga med klepetom načeloma ne more spreminjati (razen preko neposredne manipulacije URL-ja). Uporabnik se lahko odjavi s klikom na povezavo \texttt{log out}, ki stran naloži brez vseh parametrov.

\end{enumerate}

\end{document}
